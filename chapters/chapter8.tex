\documentclass[./main.tex]{subfiles}
\begin{comment}
    Conclude your work with findings
    Maintain the chronology of objectives/methodology/result
    Contribution also need to be highlighted and emphasized here
    
1) DISCUTIR Dessauges2017 para ver preguntas para el futuro!
2) DISCUTIR SHANKARAN
3) DISCUTIR Casani-Galdon 2022
4) https://sci-hub.se/10.1063/1.4962326 esto para ver modelos "" resumidos"" y por qué sirven
\end{comment}
\begin{document}

%**** esto lo traje de introducción ***
%La dinámica de activación de ERK en escalas temporales cortas y con estimulación continua del receptor del factor de crecimiento epidérmico (EGF) - un receptor de tirosina quinasa que regula la actividad de ERK en muchas células más diferenciadas que las ESCs- se estudió en otros tipos celulares y muestra comportamientos muy diversos. En muchos tipos celulares, pero no en todos, se reportó que la actividad de ERK era pulsátil, y que dicha actividad pulsátil era estocástica \cite{Aoki2013}. En otros tipos celulares, se reportó que la actividad de ERK era oscilatoria \cite{Shankaran2009}, o simplemente se observaron pulsos discretos \cite{Albeck2013}.


%En varios casos, la frecuencia de los pulsos de actividad ERK depende de la concentración de EGF o de la densidad celular \cite{Albeck2013,Aoki2013}. Esto ha llevado a sugerir que la red RAS/RAF/MEK/ERK, cuya actividad es desencadenada por la activación del receptor de EGF, codifica la información sobre los niveles de la señal extracelular con una frecuencia modulada \cite{Albeck2013}. En cambio, en otros tipos celulares, los pulsos de translocación nuclear de ERK tienen una frecuencia constante en un rango de niveles de estimulación de EGF \cite{Shankaran2009}. 



%intro biológica
La diferenciación de las células del embrión de preimplantación de ratón y de las células madre embrionarias depende de la red de señalización FGF/ERK, principalmente estimulada por la señalización de FGF4 que producen las mismas células. Las células madre embrionarias son un sistema modelo manejable que recapitula estas características. Se ha mostrado que la señalización de ERK es altamente dinámica en otros contextos biológicos. A pesar de que las funciones de FGF/ERK durante la diferenciación en ESCs son conocidas, su dinámica de señalización en este contexto de desarrollo está poco estudiada. En particular, la dinámica de señalización de ERK en escalas temporales cortas estaba hasta el momento inexplorada. Desde un enfoque interdisciplinario, en esta tesis nos propusimos realizar una descripción cuantitativa y teórica de la dinámica de señalización celular de la vía FGF/ERK, que busca ser un aporte para revelar los mecanismos de regulación de la red de señalización y/o predecir las respuestas celulares. 


%preguntas
%Con un enfoque interdisciplinario, en esta tesis nos preguntamos, a nivel celular, (i) cómo es la dinámica de activación de ERK en células madre embrionarias; (ii) cuáles de sus rasgos dinámicos constituyen información sobre los estímulos extracelulares y las características, tanto definitorias como transitorias, del tipo celular. Además, a nivel poblacional, (iii) qué tan homogénea es la respuesta de activación de ERK en una población de células en estos distintos contextos. Estas preguntas nos conducen a realizar una descripción cuantitativa y teórica de la dinámica de señalización celular de la vía FGF/ERK, que busca ser un aporte para revelar los mecanismos de regulación de la red de señalización y/o predecir las respuestas celulares. 


%chapter 2
Comenzamos explorando cómo es la dinámica de activación de ERK en células madre embrionarias. Utilizamos un sensor de traslocación para medir la dinámica de la actividad de la ERK en ESCs vivas e individuales en serum + LIF, condiciones de cultivo que mantienen la pluripotencia. Diseñamos el experimento para detectar dinámica en escalas temporales cortas, del orden del minuto, y descubrimos que la actividad de ERK es pulsátil en ESCs. Ante este resultado, quisimos conocer cuál es la estructura subyacente de los pulsos de actividad de ERK. Las estrategias tradicionales de análisis de oscilaciones sugirieron signos oscilatorios no estacionarios, pero no lograron capturar sus principales aspectos dinámicos satisfactoriamente. Desarrollamos un nuevo protocolo de análisis de series temporales que permitió dar con la primera caracterización cuantitativa de la dinámica de activación de ERK en ESCs en escalas temporales cortas. Detectamos trenes de pulsos consecutivos largos que son inconsistentes con escenarios simples de pulsado estocástico u oscilaciones. En su lugar, proponemos que estos datos pueden interpretarse como un nuevo tipo de dinámica previamente no descripta. A esta dinámica la llamamos oscilaciones intermitentes, y consiste en intervalos de oscilaciones alternados con intervalos de silencio y pulsos aislados. Esto contrasta con los pulsos estocásticos más irregulares reportados en otras líneas celulares \cite{Albeck2013,Aoki2013,Goglia2020}. Esta dinámica de señalización tiene un intervalo de interpulsado modal de $\sim$7 min (es decir, unos 8 pulsos/h), y es más rápida que la mayoría de las observadas hasta el momento.


%chapter 3
Como continuación, nos propusimos entender cómo cambian las principales propiedades de las oscilaciones intermitentes de actividad de ERK ante perturbaciones experimentales. Primero, estudiamos  cómo se modifica la dinámica de activación de ERK en ESCs ante exponer a la célula a distintas concentraciones controladas de FGF4, ligando que promueve la diferenciación celular. Realizamos mediciones de la actividad de ERK sobre células mutantes que no producían FGF4, a las que estimulamos con distintas concentraciones de ligando. Encontramos que las oscilaciones son desencadenadas ante el estímulo de FGF4. Además, están controladas por la dosis de FGF4, donde la duración de los intervalos oscilatorios aumenta con la dosis de ligando. Por otro lado, la duración de pulsos pareciera conservarse ante distintas dosis de estímulo extracelular, al igual que los valores modales del intervalo de interpulsado. 

% otros trabajos - codificación en frecuencia
En los sistemas celulares que muestran pulsaciones estocásticas de ERK, el aumento de los niveles de ligando conduce a intervalos de interpulsado más cortos y, por lo tanto, a un aumento de la tasa de pulsado \cite{Albeck2013}. Esto se ha interpretado como una codificación en frecuencia de la concentración de ligando \cite{Li2019}. Nuestras observaciones de valores modales de IPI y duraciones de pulsos conservadas nos sugieren que es poco probable que se aplique el modelo de codificación modulada por frecuencia propuesto para los modelos de pulsados estocásticos de ERK en otros tipos celulares.

% dónde surgen las oscilaciones% negative feedback
En paralelo, que las estimulaciones en ESCs con FGF4 producido por las propias células y con FGF4 extracelular en ESCs mutantes desencadene actividad oscilatoria de ERK con escalas temporales similares indica que las oscilaciones surgen en la red de transducción de señales intracelulares, de forma similar a la situación en otras líneas celulares \cite{Sparta2015}. Las escalas temporales cortas de la actividad oscilatoria nos sugieren que éstas son impulsadas por retroalimentaciones retardadas de corta escala temporal, como las modificaciones postraduccionales a nivel del receptor, o en varios niveles dentro de la cascada de señalización  \cite{Sparta2015,Lake2016,Lemmon2016,Casani2022}. La actividad oscilatoria de ERK que detectamos indica la presencia de mecanismos de retroalimentación negativa en la vía de señalización de FGF en las ESCs \cite{Novak2008}. La retroalimentación negativa en la cascada RAF/RAS/MEK/ERK establece un determinado rango de respuesta a la dosis del ligando, y linealiza la transducción de la señal a pesar de su amplificación no lineal \cite{Sturm2010}. En las ESCs y en las células del embrión temprano, la concentración de FGF4 regula la proporción de tipos de células diferenciadas. Por lo tanto, las oscilaciones de ERK en las ESCs podrían ser consecuencia de mecanismos de retroalimentación negativa que han evolucionado para sintonizar el rango de respuesta del sistema de transducción de señales con el rango fisiológicamente relevante de la concentración de FGF4 producido por las propias células. En el futuro, identificar estos posibles circuitos de retroalimentación relevantes requiere combinar estímulos temporales con perturbaciones de los candidatos y lecturas dinámicas de células únicas \cite{Blum2019}.


Para continuar, buscamos conocer qué aspectos de la dinámica de ERK son propios del estadío de diferenciación de las ESCs. Comparamos las dinámicas de activación de ERK en ESCs y EpiSCs, células levemente más diferenciadas, ambas cultivadas en FAX. Encontramos que la señalización de ERK opera en un régimen dinámico similar en ambos tipos celulares, pero que los intervalos oscilatorios son más cortos en las EpiSCs que en las ESCs, y la duración de los pulsos es más larga. Por otro lado, encontramos que los pulsos de ERK eran más rápidos cuando las ESCs crecían en FAX en lugar de en serum + LIF, lo que sugiere que los componentes del medio pueden afectar a las características de la dinámica de actividad de ERK.


%chapter 4
Observamos que la actividad pulsátil de ERK en ESCs era variable entre células cultivadas en serum + LIF, y vimos que esta variabilidad se mantenía cuando estimulábamos a las células con dosis controladas de FGF4, e incluso en las EpiSCs. Evaluamos si las células individuales tenían transiciones entre estados oscilatorios activos e inactivos en escalas temporales largas en comparación con las mediciones, lo que explicaría nuestras observaciones. Desarrollamos nuevos protocolos experimentales para adquirir la señal de actividad de ERK durante todo el ciclo celular. Estas nuevas series temporales eran más largas, pero de menor resolución. Implementamos un nuevo protocolo de análisis que nos permitió estudiar cómo cambia la actividad de ERK a lo largo del ciclo celular, y distinguir estos cambios de posibles errores experimentales. Encontramos que las ESCs son más propensas a pulsar en fases más tempranas del ciclo celular. Sin embargo, esta característica no fue suficiente para explicar la heterogeneidad observada. 


Previamente se reportaron oscilaciones regulares de importación y exportación nuclear de ERK tras la estimulación de EGF en células epiteliales mamarias \cite{Shankaran2009}. En estas células, la frecuencia de las oscilaciones de ERK es insensible a los niveles de ligando en un amplio rango, similar a nuestras observaciones al estimular ESCs mutantes con FGF4. Sin embargo, las poblaciones de ESCs contienen una mezcla de células que oscilan, células que no oscilan, y células que transitan entre estos regímenes en un amplio rango de niveles de ligando. Las oscilaciones intermitentes generadas por la organización específica del tipo celular del sistema de señalización FGF/ERK introducen una fuente de heterogeneidad celular que puede ser relevante para las decisiones del destino celular en las células pluripotentes. En el embrión, la señalización FGF/ERK regula la diferenciación tanto del endodermo primitivo como de las células del epiblasto. Diferentes actividades dinámicas de señalización pueden subyacer al establecimiento de estos dos linajes discretos a partir de un tipo celular precursor común en respuesta a la misma señal \cite{Pokrass2020}. En los cultivos de ESCs, la dinámica de señalización heterogénea que describimos añade otra dimensión a las heterogeneidades transcripcionales que prefiguran la diferenciación celular \cite{Canham2010,Chambers2007,Hayashi2008,Singh2007,Toyooka2008}. Aunque la señalización no está necesariamente desactivada cuando ERK no pulsa, las oscilaciones intermitentes de ERK en las ESCs recuerdan a la dinámica transcripcional de muchos genes, para los que se alternan ráfagas de expresión y períodos de silencio \cite{Tunnacliffe2020}. Será necesario correlacionar la dinámica de señalización con los programas transcripcionales del desarrollo para discernir cómo se interrelacionan estos dos niveles y cómo se relacionan con la diferenciación celular.


%previo
Motivados por estas ideas, buscamos dar con una descripción teórica capaz de formalizar las propiedades más relevantes de la descripción conceptual de las oscilaciones intermitentes de actividad de ERK que desarrollamos. Elegimos enfocarnos en una descripción de baja dimensionalidad, que tiene la ventaja de delinear con claridad el tipo de sistema dinámico que da lugar a nuestras observaciones. Además, nos permite desarrollar una interpretación sobre el origen dinámico de las oscilaciones intermitentes de actividad de ERK, entender cuán robusta es la dinámica ante distintas perturbaciones y predecir cambios. A su vez, es posible hipotetizar cuál es el origen de la actividad oscilatoria en la red de transducción FGF/RAS/MEK/ERK -hipótesis que pueden ser testeadas con nuevos experimentos- y como es una descripción simple puede eventualmente conectarse con predicciones sobre el destino celular \cite{Dessauges2017,Casani2022}. Asimismo, una descripción de baja dimensionalidad de las oscilaciones intermitentes de actividad de ERK en ESCs es un aporte interesante desde el enfoque puramente teórico, dado que representa un tipo de dinámica previamente no descripta. En el futuro, desarrollar una descripción que integre explícitamente cada uno de los elementos que componen la red de transducción de señales FGF/RAS/RAF/MEK/ERK nos aportaría información más accesible y complementaria sobre cuál es el origen de las oscilaciones y los cambios en la dinámica de ERK. 


%capitulo 5 (leer las conclusiones que están piolas) %capitulo 6
Comenzamos por estudiar las propiedades dinámicas de un modelo de fase con bifurcación de ciclo infinito de baja dimensionalidad, en donde un ciclo representa un pulso de ERK. Regulando la amplitud de modulación, este modelo puede dar lugar a transiciones entre un régimen oscilatorio y uno excitable, que puede dar lugar a actividad pulsátil si es perturbado. Los procesos dentro de la célula son fundamentalmente estocásticos y, como la célula es un sistema que opera fuera del equilibrio, las fluctuaciones no siempre pueden ser despreciadas e incluso desempeñan un papel fundamental \cite{Tsimiring2014}. Con esta motivación, estudiamos los efectos de perturbar al modelo determinista de manera sistemática añadiendo ruido blanco gaussiano aditivo. En el régimen excitable, el ruido promueve la actividad pulsátil y valores intermedios de ruido maximizan la coherencia de la actividad pulsátil. En el régimen oscilatorio, las oscilaciones pierden su regularidad a medida que crece el ruido. Estas características convertían a esta descripción teórica en un candidato interesante para describir las oscilaciones intermitentes, y fue la base de nuestro desarrollo teórico. Además, en sistemas lejos del equilibrio, el ruido está presente inevitablemente, y sus orígenes pueden ser muy diversos \cite{Lindner2004}. En particular, el ruido puede estar presente en sistemas excitables como reacciones químicas, lásers, la dinámica del clima, neuronas, e incluso en redes de transducción de señales intracelulares como las que regulan la dinámica de ERK es ESCs. Cómo el ruido afecta a sistemas excitables es una pregunta relevante para muchas áreas del conocimiento, y presenta muchas aplicaciones.

Logramos obtener una expresión teórica para la duración de pulsos en el régimen excitable del modelo determinista, y encontramos que la duración de pulsos depende de la relación entre la amplitud de modulación y la frecuencia del uniforme, al igual que en el caso oscilatorio, y de la magnitud de la excitación que da lugar al pulso. Observamos que la duración de pulsos diverge en donde ocurre la bifurcación tanto en el régimen excitable como en el oscilatorio, así como cuando las perturbaciones son menores o iguales que el umbral de excitabilidad. Mediante el formalismo de tiempo de primer pasaje, obtuvimos una expresión analítica de la duración de pulsos media para el modelo con ruido. Observamos que la duración de pulsos disminuye a medida que aumenta el ruido, y converge a una duración límite finita para valores altos de ruido, independientemente del valor del resto de los parámetros del modelo. Además, existe una discontinuidad en la duración de pulsos en función de la amplitud de modulación en donde ocurre la bifurcación de la descripción determinista. El comportamiento en el entorno de este punto tiende a comportarse según el resultado determinista para valores muy pequeños de ruido. Asimismo, mientras que para el caso excitable la duración siempre decrece, si la duración crece o decrece en el régimen oscilatorio depende del resto de los parámetros del modelo. Este resultado sugiere que existen parámetros para los cuales en transiciones entre el régimen oscilatorio y el excitable generadas a través de variaciones en la amplitud de modulación la duración de pulsos media podría conservarse. Alternativamente, pequeñas variaciones de los parámetros cerca de la bifurcación podrían dar lugar a variabilidad en la duración de pulsos consistentes con el ancho de las distribuciones de duración de pulsos que observamos en los experimentos. Además, la variabilidad que posiblemente aporte el carácter estocástico de las perturbaciones del modelo que proponemos también podría aportar a reproducir este aspecto de nuestras observaciones. 


A partir del modelo de fase con bifurcación de ciclo infinito y ruido blanco gaussiano aditivo, establecimos un diálogo entre experimentos y teoría entender si con esta descripción era posible reproducir la dinámica observada en los experimentos, o si era necesario proponer modificaciones para incorporar aspectos de la dinámica de actividad de ERK que el modelo no podía reproducir. Analizamos, primero, la posibilidad que el esta descripción ajuste a la estadística de duración de pulsos, al intervalo de interpulsado y la tasa de pulsado de los experimentos. Observamos que la estadística de estas tres medidas calculadas sobre series temporales simuladas eran compatibles con los valores adquiridos experimentalmente. Luego, diseñamos un protocolo basado en el método ABC-SMC para ajustar de manera sistemática los observables calculados sobre el modelo teórico a los experimentales, y evaluamos la capacidad de este ajuste de reproducir las oscilaciones intermitentes de actividad de ERK. Comparamos la estadística de magnitudes dinámicas que describían las oscilaciones intermitentes que surgían de series sintéticas simuladas a partir de parámetros que obtuvimos como resultado del ajuste con las de los experimentos. Estas magnitudes fueron (i) la cantidad de pulsos totales, aislados y consecutivos, (ii) la proporción de tiempo que las células pulsaban, y (iii) la frecuencia de trenes de pulsos de determinada duración. Observamos la mayoría de los pulsos eran aislados o se organizaban en en trenes cortos en comparación con los experimentos, y la actividad pulsátil más homogénea. 


Como estos resultados no fueron satisfactorios, propusimos modificaciones al modelo teórico. En trabajos previos, se ha descripto un posicionamiento cercano a la transición entre el comportamiento oscilatorio y uno no oscilatorio en otros tipos celulares \cite{Camalet2000,Eguiluz2000,Westendorf2013,Webb2016}. En este escenario, el estado no oscilatorio podría ser un régimen excitable que produce pulsos aislados \cite{MartinezCorral2018,Monke2017}, o los pulsos aislados podrían ser el resultado de breves excursiones desde el régimen estacionario al oscilatorio. Comenzamos por explorar la idea de que la actividad pulsátil de ERK pueda ser descripta a partir de transiciones entre el régimen oscilatorio y el excitable, que sin perturbaciones daba lugar a intervalos de silencio. Estudiamos los efectos de proponer fluctuaciones lentas en la amplitud de modulación a través de describir la amplitud de modulación sea descripta como un proceso de OU. La nueva propuesta fue capaz de recapitular la heterogeneidad de actividad pulsátil de nuestras observaciones, pero sobrestimaba los consecutivos y subestimaba los aislados, a diferencia del modelo anterior con ruido blanco gaussiano aditivo. Esta diferencia se tradujo en sistemáticamente una mayor cantidad de trenes largos de pulsos que nuestras observaciones experimentales. Como alternativa, propusimos un sistema dinámico que simultáneamente diera lugar a transiciones entre los regímenes dinámicos oscilatorio y excitable, y tuviera perturbaciones capaces de generar actividad pulsátil en el excitable. Analizamos los efectos de proponer un modelo de fase con bifurcación de ciclo infinito con la amplitud de modulación como un proceso de OU, y ruido blanco gaussiano aditivo. Encontramos que esta descripción teórica reproducía correctamente las principales características que describen las oscilaciones intermitentes. Con el modelo propuesto logramos recapitular la cantidad de pulsos totales y aislados del experimento, lo que sugiere que es necesario excitar el régimen excitable para describir la estadística de pulsos aislados de nuestras observaciones. Además, reprodujimos satisfactoriamente la tendencia de la frecuencia de de trenes de pulsos de hasta 8 pulsos de duración, lo que indica que es necesario tener transiciones entre el régimen excitable y el oscilatorio para reproducir la estadística de trenes de pulsos largos que observamos en los experimentos. Finalmente, reprodujimos satisfactoriamente tanto la heterogeneidad de la actividad pulsátil de nuestras observaciones, como la proporción de tiempo en que las células pulsaban en promedio. Incluso fue posible generar oscilaciones intermitentes con escalas temporales de duración de pulsos e intervalo de interpulsado similares a las que medimos en los experimentos. 


Observamos que el modelo fallaba en reproducir la presencia de trenes de 9 o más pulsos, y una leve subestimación de la cantidad de pulsos consecutivos. Además, subestimaba los valores altos de actividad pulsátil de células únicas. Este comportamiento sí fue reproducido con el modelo donde la amplitud de modulación era un proceso de OU, pero no había ruido blanco gaussiano. Adjudicamos que el origen de las leves discrepancias a que estas son regiones límite en donde la evolución de los datos experimentales parecía no responder a la tendencia que se registra en cada análisis. Este cambio de comportamiento de los datos experimentales podría ser un genuino comportamiento de la actividad dinámica de ERK, o podría ser consecuencia de limitaciones en las mediciones experimentales. Para discriminar entre estas dos opciones, sería conveniente a futuro realizar nuevas mediciones experimentales de igual o levemente menor frecuencia de adquisición, pero de mayor longitud.

%% si! - novedad , discusion

Para terminar, evaluamos si las fluctuaciones en la amplitud de modulación podían ser estacionarias. Consideramos un nuevo modelo de fase de bifurcación de ciclo infinito con ruido blanco gaussiano aditivo, y donde cada medición tenía una amplitud de modulación positiva muestreada de una distribución gaussiana. Encontramos que esta nueva propuesta reproducía simultáneamente la cantidad de pulsos totales, y la heterogeneidad de nuestras observaciones de actividad pulsátil y la promedio. Para reproducir la heterogeneidad de actividad pulsátil que observamos en los experimentos parece no ser necesario una escala temporal que regule la variabilidad de la amplitud de modulación. Sin embargo, el modelo subestimaba la cantidad de pulsos consecutivos y sobrestimaba los aislados, subestimando también la frecuencia de trenes de pulsos largos. Esta discrepancia sugiere que para reproducir las oscilaciones intermitentes de actividad de ERK en ESCs es necesario incorporar una escala temporal que regule las transiciones del sistema dinámico entre los regímenes oscilatorio y el excitable. 


Por primera vez, logramos describir las oscilaciones intermitentes de actividad de ERK en ESCs. Con el enfoque que tomamos para construirla, esta descripción de la dinámica de activación de ERK tiene los ingredientes mínimos para reproducir las oscilaciones intermitentes que observamos en los experimentos. Esta descripción propone transiciones de naturaleza estocástica entre un régimen oscilatorio y uno excitable reguladas por una escala temporal, y excitaciones en el régimen excitable que da lugar a actividad pulsátil. A partir de estas observaciones, se abren algunos interrogantes. Por ejemplo, ¿de qué parámetros depende la longitud de los intervalos oscilatorios, o la tasa de pulsado?, ¿es posible variar estas cantidades pero mantener la duración de pulsos constante?, o ¿esta descripción teórica es suficiente para describir cómo decae la actividad pulsátil a lo largo del ciclo celular? Responder este tipo de preguntas es clave para describir cómo depende la dinámica de activación de ERK de FGF en ESCs, y entender cómo la red de transducción de señales FGF/ERK codifica información importante para las decisiones de destino celular. A partir de nuestros resultados, interpretamos que el sistema de transducción de señales FGF/ERK en las ESCs está organizado en las proximidades de un punto de transición entre un estado no oscilatorio y uno oscilatorio. En este caso, el aumento de los niveles de FGF4 acercaría el sistema a este punto. En nuestra descripción teórica, el aumento de FGF4 podría representarse como una disminución en el valor medio amplitud de modulación, que desemboque en transiciones más frecuentes y largas al régimen oscilatorio. Del mismo modo, que la actividad pulsátil de ERK decaiga a lo largo del ciclo celular -posiblemente originada a través de cambios en la relación entre su superficie y volumen que ocurren en la célula a lo largo de su ciclo o en la expresión dependiente del ciclo celular de los componentes del sistema de señalización FGF/ERK- puede interpretarse como un alejamiento de las células del estado oscilatorio a uno no oscilatorio. En nuestra descripción teórica esto podría describirse a partir de que las fluctuaciones lentas produzcan un aumento en la amplitud de modulación a lo largo del ciclo celular. Para evaluar estas hipótesis o responder a estas preguntas creemos necesario perfeccionar el protocolo con el que ajustamos los parámetros del modelo, y el algoritmo de detección de pulsos que implementamos para las series temporales simuladas.


 %Algunas hipótesis que surgen de integrar nuestro análisis teórico con los resultados experimentales son, por ejemplo, que el aumento de la frecuencia de trenes de pulsos largos que observamos en los experimentos cuando aumentábamos la dosis de estímulo FGF4 podría ser producto de una disminución en el valor medio amplitud de modulación, que desemboque en transiciones más frecuentes y largas al régimen oscilatorio. Además, 

%Las células madre embrionarias constituyen un sistema experimental interesante para estudiar el desarrollo embrionario, y además son una importante promesa en áreas como la medicina regenerativa, el descubrimiento de fármacos y la ingeniería de tejidos \cite{Waisman2019}. Realizar una descripción cuantitativa y teórica de la dinámica de señalización celular de la vía FGF/ERK busca ser un aporte para revelar los mecanismos de regulación de la red de señalización y/o predecir las respuestas celulares \cite{Shankaran2009}. 


%La comprensión los mecanismos que gobiernan los procesos de diferenciación celular es crucial para desarrollar técnicas de diferenciación celular dirigida, técnicas sumamente útiles en el desarrollo de la medicina regenerativa, en la investigación de enfermedades o en el monitoreo y testeo de drogas experimentales \cite{Keller2005,Inoue2014}. Por otro lado, estos son conocimientos relevantes para desarrollar nuevas tratamientos en los casos en que falla la elección del destino celular. Debido a su papel central en las decisiones de destino celular, la desregulación de la red de ERK conduce a varias enfermedades, como por ejemplo el cáncer \cite{Dessauges2022,Bugaj2018,Lavoie2020,Grieco2013}, y estos resultados podrían ser relevantes para desarrollar tratamientos contra esas enfermedades. Como complemento, esperamos poder aportar nuevos conocimientos en el área de procesos estocásticos y dinámica no lineal, motivados por nuevas preguntas que surjan de la caracterización cuantitativa del sistema biológico que estudiamos y queremos describir.

%Las oscilaciones intermitentes no fueron descriptas hasta el momento. Tener un modelo matemático que describa este sistema dinámico nos permitirá entender algunos aspectos del sistema. Por ejemplo, acaso la heterogeneidad tiene origen en una exposición heterogenea a FGF de las células, o de si tiene que ver con que son mediciones cortas y no contemplan todo el ciclo celular. 

\end{document}