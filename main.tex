\documentclass[graybox,envcountchap,sectrefs]{svmono}

\usepackage{type1cm}         

\usepackage{makeidx}         % allows index generation
\usepackage{graphicx}        % standard LaTeX graphics tool
                             % when including figure files
\usepackage{multicol}        % used for the two-column index
\usepackage[bottom]{footmisc}% places footnotes at page bottom

\usepackage{newtxtext}       % 
\usepackage{newtxmath}       % selects Times Roman as basic font

%\usepackage{mathptmx}
%\usepackage{amsmath} 
%\usepackage{amssymb}
%\usepackage{times}
\usepackage{marginnote}

%%%%%%%%%%%%%%%%%%%%%%%%%%%%%%%%%%%%%%%%%%%%%%%%%%%%%%%%%%%%%%%%%%%%%
\usepackage{xr}
\usepackage[utf8]{inputenc} 
\usepackage[spanish]{babel}
\usepackage{csquotes}% no se para que es
\usepackage{setspace} %para la carátula
\usepackage{fancyhdr}

\usepackage[hidelinks]{hyperref}
\usepackage{chngpage} 
\usepackage{xspace} 
\renewcommand{\theenumi}{\alph{enumi}}
\renewcommand{\labelenumi}{{\theenumi})} 
\usepackage{comment}
%%%%%%%%%%%%%%%%%%%%%%%%%%%%%%%%%%%%%%%%%%%%%%%%%%%%%%%%%%%%%%%%%%%%%

%%% bibliografía
\usepackage[backend=biber, maxnames=15, url=false, doi=false, isbn=false, citestyle=numeric-comp, bibstyle=ieee, sorting=none, date=year]{biblatex} %maxalphanames=1 
%\DefineBibliographyStrings{spanish}{andothers={\textit{et al.}}}
%\renewbibmacro{in:}{}
%\renewcommand*{\labelalphaothers}{}

%\DeclareLabelalphaTemplate{
%  \labelelement{
%    \field[final]{shorthand}
%    \field{labelname}
%    \field{label}
%  }
%  \labelelement{
%    \field[strwidth=4, strside=right]{year}
%  }
%}
\addbibresource{./autora/referencias.bib}
\providecommand{\e}[1]{\ensuremath{\times 10^{#1}}}

%\oddsidemargin -0.25in		% Left margin is 1in + this value
%\textwidth 6.75in		% Right margin is not set explicitly
%\topmargin 0in			% Top margin is 1in + this value
%\textheight 9in			% Bottom margin is not set explicitly

\usepackage[affil-it]{authblk} 


\newcommand*\mybackmatter{%
\startcontents
\phantomsection
\addcontentsline{toc}{part}{}%
\@endpart}

%%%%%%%%%%%%%%%%%%%%%%%%%%%%%%%%%%%%%%%%%%%%%%%%%%%%%%%%%%%%%%%%%%%%%

% see the list of further useful packages
% in the Reference Guide

\makeindex             % used for the subject index
                       % please use the style svind.ist with
                       % your makeindex program

%%%%%%%%%%%%%%%%%%%%%%%%%%%%%%%%%%%%%%%%%%%%%%%%%%%%%%%%%%%%%%%%%%%%%
\graphicspath{{./}{figures/}}

%%%%%%%%%%%%%%%%%%%%%%%%%%%%%%%%%%%%%%%%%%%%%%%%%%%%%%%%%%%%%%%%%%%%%

\usepackage{appendix}
\AtBeginEnvironment{subappendices}{%
\chapter*{Apéndices}
\addcontentsline{toc}{chapter}{Apéndices}
%\counterwithin{figure}{section}
%\counterwithin{table}{section}
}
%%%%%%%%%%%%%%%%%%%%%%%%%%%%%%%%%%%%%%%%%%%%%%%%%%%%%%%%%%%%%%%%%%%%%

%% paquetes para la nomenclatura
%\usepackage{nomencl}
%% This code creates the groups
% -----------------------------------------
%\usepackage{etoolbox}
%\renewcommand\nomgroup[1]{%
%  \item[\bfseries
%  \ifstrequal{#1}{V}{Variables}{%
%  \ifstrequal{#1}{A}{Abreviaciones}}%
%]}
% -----------------------------------------

\DeclareMathOperator\erf{erf} 

%%% las variables del pulse detection
\newcommand*{\epsplus}{\ensuremath{\epsilon_{+}}\xspace}
\newcommand*{\xxi}{\ensuremath{\theta_i}\xspace}
\newcommand*{\xxe}{\ensuremath{\theta_e}\xspace}
\newcommand*{\xx}{\ensuremath{\theta}\xspace}
\newcommand*{\deltax}{\ensuremath{\delta\theta}\xspace}
\newcommand*{\tplus}{\ensuremath{t_{+}}\xspace}
\newcommand*{\epstplus}{\ensuremath{\epsilon t_+}\xspace}
\newcommand*{\Depstplus}{\ensuremath{\epsilon t_+ '}\xspace}
\newcommand*{\DDepstplus}{\ensuremath{\epsilon t_+ ''}\xspace}
\newcommand*{\DepstplusH}{\ensuremath{\epsilon t_{+h} '}\xspace} 
\newcommand*{\DDepstplusH}{\ensuremath{\epsilon t_{+h} ''}\xspace} 
\newcommand*{\ddelta}{\ensuremath{\frac{\alpha}{\omega}}\xspace}
\newcommand*{\dddelta}{\ensuremath{\alpha/\omega}\xspace}
\newcommand*{\VD}{\ensuremath{\frac{V(x)}{D}}\xspace}

\linespread{1.2} %espacio entre lineas
%\makenomenclature
\usepackage{subfiles}
\begin{document}


\frontmatter%%%%%%%%%%%%%%%%%%%%%%%%%%%%%%%%%%%%%%%%%%%%%%%%%%%%%%
\subfile{autora/A0_caratula}

%\newpage
%\author{Lic. Fiorella Fabris}
%\title{Oscilaciones intermitentes de señales moleculares\\en células madre embrionarias}
%\subtitle{un enfoque interdisciplinario}
%\date{11 de Agosto de 2022}
%\maketitle

%\subfile{autora/A1_dedicatoria}
\subfile{autora/A1_agradecimientos}
\subfile{autora/A2_resumen}
\subfile{autora/A3_nota_de_autora}
\tableofcontents
\subfile{autora/A4_nomenclatura}


\mainmatter%%%%%%%%%%%%%%%%%%%%%%%%%%%%%%%%%%%%%%%%%%%%%%%%%%%%%%%
\pagenumbering{arabic}

\chapter{Introducción}
\label{ch1}
\subfile{chapters/chapter1}

\chapter{Oscilaciones intermitentes en células madre embrionarias}
\chaptermark{Oscilaciones intermitentes en ESCs}
\label{ch2}
\subfile{chapters/chapter2}

\chapter{Las oscilaciones de ERK dependen del estímulo extracelular y del estadío de diferenciación celular}
\chaptermark{Las oscilaciones de ERK dependen de ...}
\label{ch3}
\subfile{chapters/chapter3}

\chapter{Los pulsos de ERK son más prevalentes al principio del ciclo celular}
\chaptermark{Los pulsos de ERK son más prevalentes ...}
\label{ch4}
\subfile{chapters/chapter4}

\chapter{Un modelo de fase con bifurcación de ciclo infinito con ruido blanco gaussiano presenta oscilaciones, silencios y pulsos aislados}
\chaptermark{Modelo de fase con ruido blanco gaussiano...}
\label{ch5}
\subfile{chapters/chapter5}


%\chapter{Modelo de fase con bifurcación de ciclo infinito con ruido blanco gaussiano}
%\chaptermark{Modelo con ruido blanco gaussiano}
%\subfile{chapters/chapter6}
%\label{ch6} 


\chapter{Transiciones entre oscilaciones y excitabilidad describen las oscilaciones intermitentes}
\chaptermark{Transiciones entre oscilaciones y excitabilidad...}
\label{ch7}
\subfile{chapters/chapter7}

\chapter{Conclusiones generales, discusión y perspectivas}
\subfile{chapters/chapter8}
\label{ch8}

\appendix
\addtocontents{toc}{\protect\setcounter{tocdepth}{0}}
\subfile{chapters/apendice}



\backmatter%%%%%%%%%%%%%%%%%%%%%%%%%%%%%%%%%%%%%%%%%%%%%%%%%%%%%%%

\cleardoublepage
\addcontentsline{toc}{chapter}{Bibliografía}
\printbibliography

\end{document}



\mainmatter%%%%%%%%%%%%%%%%%%%%%%%%%%%%%%%%%%%%%%%%%%%%%%%%%%%%%%%
%\include{author/part}
%\include{author/chapter}
%\include{author/appendix}

%\backmatter%%%%%%%%%%%%%%%%%%%%%%%%%%%%%%%%%%%%%%%%%%%%%%%%%%%%%%%
%\include{author/glossary}
%\include{author/solutions}
%\printindex

%%%%%%%%%%%%%%%%%%%%%%%%%%%%%%%%%%%%%%%%%%%%%%%%%%%%%%%%%%%%%%%%%%%%%%

\end{document}

