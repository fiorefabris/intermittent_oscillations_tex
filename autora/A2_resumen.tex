\documentclass[./main.tex]{subfiles}


\begin{document}

\Extrachap{Resumen}

Si bien se conoce que la diferenciación de las células del embrión de preimplantación de ratón y de las células madre embrionarias (ESCs) requiere de la activación de ERK, su dinámica de señalización en escalas temporales cortas no estaba explorada en este contexto biológico. Desde un enfoque interdisciplinario, en esta tesis construimos la primera descripción cuantitativa y teórica de la dinámica de activación de ERK en ESCs en escalas temporales cortas. 

Primero extraemos series temporales a partir de filmaciones de colonias de células que tienen integrado un sensor de translocación capaz de medir la actividad de ERK en células individuales. Con estrategias de análisis de series temporales que desarrollamos en este trabajo, encontramos que el patrón de actividad de ERK consiste en un tipo de dinámica previamente no descripto, que llamamos oscilaciones intermitentes: intervalos de oscilaciones alternados con intervalos de silencios y pulsos aislados. 

A partir de mediciones en células mutantes que no producen el principal ligando que desencadena la actividad de ERK en ESCs, encontramos que los intervalos oscilatorios son desatados por el estímulo extracelular y aumentan su duración a mayores dosis. Luego comparamos las dinámicas de activación de ERK en ESCs y un tipo celular levemente más diferenciado, y hallamos que la extensión de los intervalos oscilatorios depende del estadío de diferenciación celular. Realizamos nuevas mediciones que abarcan todo el ciclo celular en ESCs. Implementamos un nuevo análisis de series temporales más largas pero de menor resolución, y encontramos que las ESCs son más propensas a pulsar en fases más tempranas del ciclo celular.


Formalizamos la descripción cuantitativa que construimos en un modelo matemático de baja dimensionalidad. Comenzamos por introducir un modelo de fase de bifurcación de ciclo infinito y ruido blanco gaussiano, y estudiamos sus principales propiedades dinámicas mediante simulaciones numéricas y formalismos teóricos. En este análisis utilizamos el formalismo de tiempo de primer pasaje para obtener la expresión analítica de la duración de pulsos media. 

A partir de comparar los experimentos con la teoría mediante ajustes basados en la técnica de Cálculo Bayesiano Aproximado Monte Carlo Secuencial, propusimos modificaciones al modelo inicial que nos permitió construir una descripción teórica de la dinámica de activación de ERK que tiene los ingredientes mínimos para reproducir las oscilaciones intermitentes que observamos en los experimentos. Esta descripción propone transiciones de naturaleza estocástica entre un régimen oscilatorio y uno excitable reguladas por una escala temporal, y perturbaciones generadas por ruido que en el régimen excitable dan lugar a actividad pulsátil.

%Posteriormente, buscamos resumir el modelo conceptual que construimos en una descripción teórica de baja dimensionalidad. Comenzamos por un modelo de fase con bifurcación de ciclo infinito, capaz de dar lugar a oscilaciones o silencios. Le incorporamos ruido aditivo, que puede dar lugar a actividad pulsátil coherente entre silencios o desordenar las oscilaciones. Mediante el formalismo de tiempo de primer pasaje, obtuvimos la expresión analítica de la duración de pulsos media para el modelo con ruido. Este resultado, junto con análisis de series temporales que producimos a partir de simulaciones numéricas, sugerían que este modelo podría reproducir la dinámica observada en los experimentos.

%Ajustamos el modelo a los experimentos mediante el método de Cálculo Bayesiano Aproximado Monte Carlo Secuencial, y hallamos que no es capaz de reproducir las características dinámicas esenciales de las oscilaciones intermitentes. Motivados por estos resultados, proponemos añadir fluctuaciones lentas a partir de incorporar un proceso de Ornstein-Uhlenbeck al modelo. Finalmente, encontramos que esta descripción toórica tiene los ingredientes mínimos y necesarios para reproducir las oscilaciones intermitentes de activación de ERK en ESCs que observamos en los experimentos.

\vspace{0.5cm}

\textit{\textbf{Palabras claves:} células madre embrionarias, sistema de señalización FGF/ERK, series temporales, oscilaciones intermitentes, modelos de fase estocásticos.}



%%%%%%%%%%%%%%%%%%%%%%%%%%%%%%%%%%%%%%%%%%%%%%%%%%%%%%%%%%%%%%%%%%%%%%%%%%%%%%%%%%%%%%%%%%%%%%%%%%%%%%%%%%%%%%%%%%%%%%%%%
%%%%%%%%%%%%%%%%%%%%%%%%%%%%%%%%%%%%%%%%%%%%%%%%%%%%%%%%%%%%%%%%%%%%%%%%%%%%%%%%%%%%%%%%%%%%%%%%%%%%%%%%%%%%%%%%%%%%%%%%%
\extrachap{Abstract}

Although it is known that the differentiation of mouse preimplantation embryo cells and embryonic stem cells (ESCs) requires ERK activation, its short-term signalling dynamics in this biological context was previously unexplored. Taking an interdisciplinary approach, in this thesis we build the first quantitative and theoretical description of short-term ERK activity dynamics in ESCs. 


We first extracted time series from time-lapse movies of cell colonies that have a translocation sensor that measures ERK activity in single cells integrated. By time series analysis strategies that we developed in this work, we found that ERK activity consists in a novel type of dynamics, that we termed intermittent oscillations: intervals of oscillations interspersed with silent intervals and isolated pulses.

From measurements of mutant cells that cannot produce the main ligand that triggers ERK activity in ESCs, we found that the oscillatory intervals are activated by the extracellular stimulus and increase their duration at higher doses. Comparing ERK dynamical activity in ESCs and in a slightly more differentiated cell type, we found that the duration of the oscillatory intervals depends on the cell differentiation state. We performed new measurements over the entire cell cycle in ESCs. We implemented a new time series analysis protocol to study longer but lower-resolution traces, and found that ESCs are more likely to pulse earlier in the cell cycle.


We formalise the quantitative description we built in a low-dimensional mathematical model. We began by introducing an infinite-cycle bifurcation phase model with gaussian white noise, and studied its main dynamical properties by means of numerical simulations and theoretical formalisms. In this analysis we used the first-passage time formalism to obtain the analytical expression for the average pulse duration.

From comparing the experiments with theory by fittings based on the Approximate Bayesian Monte Carlo Sequential Computation method, we proposed modifications to the initial model that allowed us to build a theoretical description of the ERK dynamical activity with the minimum ingredients to reproduce the intermittent oscillations we observed in the experiments. This description proposes transitions of a stochastic nature between an oscillatory and an excitable regime regulated by a time scale, and noise-generated perturbations that in the excitable regime give rise to pulsatile activity.

\vspace{0.5cm}


\textit{\textbf{Key words:} embryonic stem cells, FGF/ERK signalling system, time series, intermittent oscillations, stochastic phase models.}

\end{document}
