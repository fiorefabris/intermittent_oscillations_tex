\documentclass[./main.tex]{subfiles}

\begin{document}
\Extrachap{Nota de la autora}

Esta tesis es el resultado de una colaboración interdisciplinaria entre el grupo de biología experimental de Christian Schröter, del Instituto Max Planck de Fisiología Molecular en Dortmund, Alemania, y el grupo de física teórica de Luis Morelli, del Instituto de Investigación en Biomedicina de Buenos Aires, en Buenos Aires, Argentina. Los experimentos fueron realizados por Dhruv Raina, un estudiante de doctorado del laboratorio de Christian, y yo estuve a cargo del análisis de datos y las descripciones teóricas motivadas por los experimentos. 


Durante todos estos años, Christian, Luis, Dhruv y yo generamos y mantuvimos una estrecha y fluida colaboración entre ambos grupos. Instituimos una dinámica \textit{de retroalimentación}, en donde los experimentos motivaron descripciones teóricas relevantes, y con estas descripciones realizamos predicciones que permitieron diseñar nuevos experimentos. Si bien mi trabajo fue teórico, durante el desarrollo del proyecto tuve interacción de primera mano con datos experimentales, y me involucré en la planificación y el diseño de los experimentos. Como contraparte, conté con la ventaja de discutir e interpretar mis resultados junto al resto de los miembros del proyecto. 


En esta tesis de doctorado busco plasmar esta metodología de trabajo tanto en su contenido como en su escritura. Escribo desde la primera persona del plural porque considero que esta tesis está constituida a partir del trabajo colectivo de todos los integrantes del proyecto que la enmarca. 


A lo largo de los primeros capítulos, introduzco y discuto algunos experimentos. Si bien yo no los realicé, estos experimentos significan una parte importante de mi trabajo y de mi aprendizaje, y son fundamentales para desarrollar las principales ideas que presento en esta tesis. La forma en que los expongo busca reflejar mi entendimiento y la manera de involucrarme en ellos. 

%Una parte importante del trabajo fue lograr establecer una comunicación fluída con profesionales de otras formaciones y experiencias, aprendizaje que busqué plasmar en esta tesis. 
\vspace{\baselineskip}

\begin{flushright}\noindent
%Buenos Aires, 15 de Agosto de 2022 \hfill {\it Fabris Fiorella}\\
{\it Fiorella Fabris}
\end{flushright}

\end{document}